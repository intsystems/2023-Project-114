\documentclass[12pt, twoside]{article}
\usepackage{jmlda}
\newcommand{\hdir}{.}

\begin{document}

\title
    {Классификация траекторий физических систем с помощью лагранжевых нейронных сетей}
\author
    {А.\,И.~Богданов, С.\,К.~Панченко, В.\,В.~Стрижов} 
\email
    {bogdanov.ai@phystech.edu; panchenko.sk@phystech.edu; strijov@phystech.edu}
\abstract{
    В работе решается задача классификации траекторий физических систем. Квазипериодическая динамика системы аппроксимируется лагранжианом, который восстанавливается по обобщённым координатам с помощью Лагранжевой нейронной сети. Показано, что для параметров таких сетей выполняется гипотеза компактности: векторы параметров, соответствующие траекториям различных классов, оказываются разделимы в своём пространстве. Проводится эксперимент на датасете PAMAP2, результаты которого подтверждают, что параметры лагранжевых нейронных сетей действительно являются информативным признаковым описанием для задачи классификации. 

\bigskip

\noindent
\textbf{Ключевые слова}: \emph {физическая система, лагранжиан, лагранжева нейронная сеть}

%\doi{-}
\receivedRus{-}
\receivedEng{-}
}

\maketitle

\section{Введение}
 
    Для моделирования динамики физических систем обычно используется Лагранжева динамика \cite{landau1976mechanics}. В соответствии с ней строится лагранжиан $L$, который определяется как разница между кинетической энергией ($T$) и потенциальной энергией ($V$) системы:

    $$L = T - V.$$
    Для получения уравнений движения нужно воспользоваться уравнением Эйлера-Лагранжа:

    $$\frac{d}{dt} \frac{\partial L}{\partial \dot{\mathbf{q}}} - \frac{\partial L}{\partial \mathbf{q}} = 0,$$
    полученным из принципа \emph{наименьшего действия}.

    Нейронные сети упрощают моделирование динамики физических систем и классификацию ее траекторий, так как для для их использования не требуются знания лагранжиана и решение сложных систем дифференциальных уравнений. 

    [2 предложения о лагранжевой сети]
    
    Лагранжева нейронная сеть (LNN) \cite{cranmer2020lagrangian} аппроксимируют обобщенный лагранжиан системы, с помощью которого, используя различные классификаторы, можно определить метку траектории исходя из предположения, что у различных меток сильно различаются коэффициенты лагранжиана.

    [гипотеза]

    [использование параметров лагранжевой сети в качестве признаков для решения задачи классификации]


    [мб этот абзац в другой раздел?]
    
    Для вычислительного эксперимента используется датасет PAMAP2 \cite{PAMAP2}, который содержит траектории движений человека во время различных активностей и их метки.

\newpage
    
\section{Постановка задачи классификации траекторий физических систем}

Задана выборка с метками из $n$ траекторий: 

    $$\{ \mathcal{D}_j, z_j\}_{j=1}^n,$$ 
    где  [список]

    %\{\mathbf{x}_i^{(j)}, \mathbf{y}_i^{(j)}\}_{i=1}^m\}_j

    %[\item[] x_i = ... - это то-то]

    %[\item[] y_i]
    
    $$\mathbf{x}_i = (\mathbf{q}_i, \mathbf{\dot{q}}_i)$$
    ~-- координаты траектории движения физической системы, 
    
    $$\mathbf{{y}}_i = \mathbf{\dot{x}}_i = (\mathbf{\dot{q}}_i, \mathbf{\ddot{q}}_i)$$
    ~-- динамика движения физической системы, $\mathbf{q}_i \in \mathbb{R}^{r \times n}$, где $r$ -- количество координат, $m$~-- длина траектории, $z_j$ ~-- метка класса.

\paragraph{Задачи регрессии восстановления лагранжиана}

    Сведём задачу моделирования лагранжиана системы к задаче регрессии. 

    Регрессионная модель [для траектории D\_j ?] выбирается из класса нейронных сетей:

    $$\mathbf{f_j} \colon (\mathbf{w}, \mathbf{X}) \to \mathbf{y},$$ 
    где $\mathbf{w} \in \mathbb{W}$~-- параметры модели, 
    $$\hat{\mathbf{y}}_j = \mathbf{f} _j(\mathbf{X}_j,\mathbf{w}) \in \mathbb{R}^{2\times r \times m}, \quad \mathbf{X}_j = \bigcup_{i=1}^m \mathbf{x}_i, \quad \mathbf{y}_j = ... $$. 

    Задача моделирования динамики системы представлена в виде задачи минимизации квадратичной ошибки: 

    $$\mathcal{L}(\textbf{w}) = \mathcal{L}( \mathbf{w} | \mathbf{X}_j, \mathbf{y}_j, ) = \| \hat{\mathbf{y}} - \mathbf{y} \|_2^2.$$



    $$\textbf{w}_j^* = \argmin_{\mathbf{w}\in\mathbb{W} }\left(\mathcal{L}(\textbf{w})\right).$$

\paragraph{Задачи классификации траекторий по лагранжианам}

    После решения задачи регрессии получаем задачу классификации:

    $$\{\textbf{w}^*_j, z_j\}_{j=1}^n,$$
    где $\textbf{w}^*_j$ ~-- коэффициенты обобщенного лагранжиана $j$-ой траектории, $z_j$ ~-- метка класса этой траектории.

    Для ее решения используются различные методы классификации, среди которых: LogisticRegression, GaussianProcessClassifier, RandomForestClassifier [перевести].

\section{Лагранжевы нейронные сети}

\paragraph{Лагранжева динамика}		

Лагранжев формализм \cite{cranmer2020lagrangian, goldstein:mechanics, class_mechanics, arnold_mechanics} моделирует физическую систему с координатами траектории $x_t = (q, \dot{q})$, которая начинается в состоянии $x_0$ и заканчивается в другом состоянии $x_1$. Определяется функционал, называющийся действием:

$$S=\int\limits_{t_{0}}^{t_{1}} L d t,$$
определяющий путь, по которому координаты $x_t$ пройдут из $x_0$ в $x_1$ в промежуток времени от $t_0$ до $t_1$. Путь минимизирует действие $S$, т.е. $\delta S = 0$. Это приводит к уравнению Эйлера-Лагранжа, определяющему динамику системы
$$\frac{d}{d t} \frac{\partial L}{\partial \dot{\mathbf{q}}}=\frac{\partial L}{\partial \mathbf{q}}$$

Ускорение каждой компоненты системы $ \ddot{\mathbf{q}}$ может быть напрямую получено из данного уравнения:

$$\begin{aligned} 
\frac{\partial}{\partial \dot{\mathbf{q}}} \frac{d L}{d t} 
&=\frac{\partial L}{\partial \mathbf{q}} \\ \frac{\partial}{\partial \dot{\mathbf{q}}}\left(\frac{\partial L}{\partial \mathbf{q}} \frac{d q}{d t}+\frac{\partial L}{\partial \dot{\mathbf{q}}} \frac{d \dot{\mathbf{q}}}{d t}\right) 
&=\frac{\partial L}{\partial \mathbf{q}} \\ \frac{\partial}{\partial \dot{\mathbf{q}}}\left(\frac{\partial L}{\partial \mathbf{q}} \dot{\mathbf{q}}+\frac{\partial L}{\partial \dot{\mathbf{q}}} \ddot{\mathbf{q}}\right) 
&=\frac{\partial L}{\partial \mathbf{q}} \\ \frac{\partial}{\partial \dot{\mathbf{q}}} \frac{\partial L}{\partial \mathbf{q}} \dot{\mathbf{q}}+\frac{\partial}{\partial \dot{\mathbf{q}}} \frac{\partial L}{\partial \dot{\mathbf{q}}} \ddot{\mathbf{q}} 
&=\frac{\partial L}{\partial \mathbf{q}} \\ \frac{\partial}{\partial \dot{\mathbf{q}}} \frac{\partial L}{\partial \dot{\mathbf{q}}} \ddot{\mathbf{q}} 
&=\frac{\partial L}{\partial \mathbf{q}}-\frac{\partial}{\partial \dot{\mathbf{q}}} \frac{\partial L}{\partial \mathbf{q}} \dot{\mathbf{q}} \\ \ddot{\mathbf{q}} 
&=\left(\frac{\partial}{\partial \dot{\mathbf{q}}} \frac{\partial L}{\partial \dot{\mathbf{q}}}\right)^{-1}\left[\frac{\partial L}{\partial \mathbf{q}}-\frac{\partial}{\partial \dot{\mathbf{q}}} \frac{\partial L}{\partial \mathbf{q}} \dot{\mathbf{q}}\right] \\ \ddot{\mathbf{q}} 
&=\left(\nabla_{\dot{\mathbf{q}} \dot{\mathbf{q}}} L\right)^{-1}\left[\nabla_{q} L-\left(\nabla_{\dot{\mathbf{q}}\mathbf{q}} L\right) \dot{\mathbf{q}}\right],
\end{aligned}$$

где гессиан $\left(\nabla_{\dot{\mathbf{q}\mathbf{q}}} L\right)_{i j}=\frac{\partial^{2} L}{\partial q_{j} \partial \dot{q}_{i}}$.

Таким образом, алгоритм моделирования динамики системы в лагранжевой динамике:
\begin{enumerate}
	\item Найти аналитические выражения для кинетической $(T)$ и потенциальной энергии $(V)$
	\item Получить лагранжиан $\mathcal{L} = T - V $
	\item Применить ограничение Эйлера-Лагранжа $\frac{d}{d t} \frac{\partial L}{\partial \mathbf{\dot{q}}} =\frac{\partial L}{\partial \mathbf{q}} $
	\item Решить получившуюся систему дифференциальных уравнений
\end{enumerate}

\paragraph{Лагранжева нейронная сеть}

В работе \cite{Cranmer2020LagrangianNN} предложено в нейронную сеть $$f: \mathbf{X} = (\mathbf{q}, \mathbf{\dot{q}}) \rightarrow \mathbf{Y}$$
 добавить априорные знания о физике системы, учитывая лагранжиан системы:
$$f: \mathbf{X} = (\mathbf{q}, \mathbf{\dot{q}}) \rightarrow \mathbf{L};$$

Т.е. ключевой идеей является параметризовать нейронной сетью лагранжиан $L$, получить выражение ограничения Эйлера-Лагранжа и обратно распространить ошибку через полученные ограничения
$$\ddot{\mathbf{q}} =\left(\nabla_{\dot{\mathbf{q}} \dot{\mathbf{q}}} L\right)^{-1}\left[\nabla_{q} L-\left(\nabla_{\dot{\mathbf{q}}\mathbf{q}} L\right) \dot{\mathbf{q}}\right],$$ 

На рисунке \ref{fig: base_vs_lnn} представлены схемы работы базового решения нейронными сетями и LNN для задачи моделирования динамики системы.На основе базового решения основаны работы \cite{NEURIPS2018_69386f6b, 712178}.

\begin{comment}
\begin{figure}[H]
	\centering
	\begin{subfigure}[b]{0.49\textwidth}
		\centering
		\includegraphics[width=0.55\textwidth]{baseline_nn_scheme.png}
		\caption{Базовое решение нейронными сетями}
		\label{fig:y equals x}
	\end{subfigure}
	\hfill
	\begin{subfigure}[b]{0.49\textwidth}
		\centering
		\includegraphics[width=\textwidth]{lnn_scheme.png}
		\caption{Решение LNN}
		\label{fig:three sin x}
	\end{subfigure}
\caption{Схемы работы базового решения нейронными сетями (a) и LNN (b) для задачи моделирования динамики физической системы}
\label{fig: base_vs_lnn}
\end{figure}
 \end{comment}

В качестве нейронной сети $f: \mathbf{X} = (\mathbf{q}, \mathbf{\dot{q}}) \rightarrow \mathbf{L}$ берется полносвязная сеть с 3-мя слоями. Таким образом, для заданных координат  $(\mathbf{q}, \mathbf{\dot{q}})$ имеем модель с априорными знанями о законе сохранения энергии, которой можем получить динамику параметров $(\mathbf{\dot{q}}, \mathbf{\ddot{q}})$.

\section{Вычислительный эксперимент}

[Цель эксперимента: сравнение качества моделирования механической системы различными нейронными сетями, качество будем сверять с аналитическим решением]

\paragraph{Данные}

\paragraph{Обработка данных}

\section{Результаты}

\section{Заключение}

\bibliographystyle{plain}
\bibliography{Bogdanov2023LNN}

\begin{comment}
    
\bibitem{Isachenko_en}
    \emph{Isachenko R.V., Strijov V.V.}
    Quadratic programming feature selection for multicorrelated signal decoding with partial least squares~//
    Expert Systems with Applications,
    Volume 207, 30 November 2022, 117967
    
\end{comment}

\end{document}
