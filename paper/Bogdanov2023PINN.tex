\documentclass[12pt, twoside]{article}
\usepackage{jmlda}
\newcommand{\hdir}{.}

\begin{document}

\title
    {Моделирование динамики физических систем с помощью Physics-Informed Neural Networks}
\author
    {А.\,И.~Богданов, С.\,К.~Панченко} 
\email
    {bogdanov.ai@phystech.edu; panchenko.sk@phystech.edu}
\abstract
    {В работе решается задача улучшения качества моделирования динамики сложных механических систем с помощью нейронных сетей. Большинство подходов, оперирующих лишь набором обобщённых координат в качестве входа, не способны предсказывать поведение системы с высокой точностью.
    В работе исследуется Нётеровская Лагранжева нейронная сеть, которая учитывает закон сохранения энергии, импульса, момента импульса и способна восстанавливать лагранжиан системы по траекториям её движения. Вместо полносвязанных слоев в сети используются модификации внутренней архитектуры, основанные на свёрточных и рекурентных нейронных сетях. Сравнение результатов моделирования проводилось на искусственно сгенерированных данных для пружинной системы. Результаты подтверждают то, что качество модели увеличивается, если она обладает информацией о системе.
\bigskip

\noindent
\textbf{Ключевые слова}: \emph {нейронная сеть; механическая система, лагранжиан}

%\doi{-}
\receivedRus{-}
\receivedEng{-}
}

\maketitle
%\linenumbers

\section{Введение}
 
    Для моделирования динамики физических систем обычно применяется Лагранжева динамика \cite{landau1976mechanics}. Чтобы её использовать, нужно построить лагранжиан $L$, который определяется как разница между кинетической энергией ($T$) и потенциальной энергией ($V$) системы:

    $$L = T - V.$$
    Для получения уравнений движения, нужно воспользоваться уравнением Эйлера-Лагранжа:

    $$\frac{\partial L}{\partial x} - \frac{d}{dt} \frac{\partial L}{\partial \dot{x}} = 0,$$
    
    которое получено из \emph{принципа наименьшего действия}.

    Нейронные сети ускорят и упростят моделирование динамики физических систем, так как для для их использования не требуется знания лагранжиана, а также нет необходимости решать сложные системы дифференциальных уравнений. Но классические нейронные сети не обладают информацией о системе, которую моделируют, поэтому используются иные.

    Лагранжевы нейронные сети (LNN) [ссылка на Лагранжевы нейронные сети] аппроксимируют лагранжиан системы, из которого с помощью уравнения Эйлера-Лагранжа находится динамика системы. В этой модели лагранжиан не зависит от времени, т.е. учитывается только закон сохранения энергии, но не учитываются другие законы сохранения.

    Нётеровские нейронные сети (NLNN) [ссылка на Пашу], которые являются модификацией Лагранжевых нейронных сетей, учитывают законы сохранения импульса и момента импульса и моделируют только потенциальную энергию системы, зависящую от разности обобщённых координат. 

    В работе предлагается исследование модификаций Нётеровской нейронной сети, изменяющих структуру слоев: гипотеза состоит в том, что использование свёрточных и рекуррентных нейронных сетей в качестве внутренних слоёв NLNN способно улучшить их качестве в применении к некоторым системам.

    Для вычислительного эксперимента была взята пружинная система. В данной системе сохраняется энергия, импульс и момент импульса. Для моделирования динамики системы были взяты полносвязанная Нётеровская LNN, свёрточная Нётеровская LNN и рекуррентная Нётеровская LNN. Результаты сравнивались с траекториями пружинной системы, полученными аналитическим решением с помощью метода Рунге-Кутты 4-го порядка [ссылка на метод].
    
\section{Постановка задачи регрессии динамики физической системы}

    Задачу моделирования динамики системы можно свести к задаче регрессии. Пусть дана выборка из $m$ траекторий: 

    $$\{\mathbf{x}_i, \mathbf{y}_i\}_{i=1}^m,$$ 
    где $\mathbf{x}_i = (\mathbf{q}_i, \mathbf{\dot{q}}_i)$~--  координаты траектории движения пружинной системы, $\mathbf{{y}}_i = \mathbf{\dot{x}}_i = (\mathbf{\dot{q}}_i, \mathbf{\ddot{q}}_i)$~-- динамика движения пружинной системы, $\mathbf{q}_i \in \mathbb{R}^{r \times n}$, где $r$ -- количество координат, $n$~-- длина траектории.

    Регрессионная модель выбирается из класса нейронных сетей:

    $$\{\mathbf{f}_k\colon(\mathbf{w}, \mathbf{X})\to  \hat{\mathbf{y}} \mid k \in \mathcal{K}\},$$ 
    где $\mathbf{w} \in \mathbb{W}$~-- параметры модели, $\hat{\mathbf{y}} = \mathbf{f} (\mathbf{X},\mathbf{w}) \in \mathbb{R}^{2\times r \times n}, \mathbf{X} = \bigcup_{i=1}^m \mathbf{x}_i$.

    В качестве функция ошибки взята квадратичная ошибка:

    $$\mathcal{L}(\mathbf{y}, \mathbf{X}, \mathbf{w}) = \| \hat{\mathbf{y}} - \mathbf{y} \|_2^2.$$

    Таким образом, задача моделирования динамики системы представлена в виде задачи минимизации квадратичной ошибки: 

    $$\textbf{w}^* = \argmin_{\mathbf{w}\in\mathbb{W} }\left(\mathcal{L}(\textbf{w})\right).$$

\paragraph{Название параграфа}
Разделы и~параграфы, за исключением списков литературы, нумеруются.

\section{Заключение}
Желательно, чтобы этот раздел был, причём он не~должен дословно повторять аннотацию.
Обычно здесь отмечают, каких результатов удалось добиться, какие проблемы остались открытыми.

\bibliographystyle{plain}
\bibliography{Bogdanov2023PINN}

\end{document}
